\section{Exercise : $b$-flow}

The figures from the assignment is re-created with $b$-flows on the edges in red

\subsection{case (a)}

\begin{figure}[h]
    \centering
    \begin {tikzpicture}[-latex, auto, node distance=24mm and 36mm ,on grid ,
                          semithick ,
                          main node/.style ={ circle , color=white ,
                          draw , black , text=black , minimum width =8mm}]
      \node[main node] (v1) [label=above left:$3$]{$v_1$};
      \node[main node] (v2) [right=of v1, label=above right:$-2$] {$v_2$};
      \node[main node] (v3) [below=of v1, label=above left:$4$] {$v_3$};
      \node[main node] (v4) [right=of v3, label=above right:$2$] {$v_4$};
      \node[main node] (v5) [node distance=12mm and 18 mm, below right=of v3, label=above:$-7$] {$v_5$};
  \path[every node/.style={font=\sffamily\small}]
    (v1) edge node [left] {\begin{color}{red}$2$\end{color}/$4$} (v3)
         edge node [above] {\begin{color}{red}$0$\end{color}/$1$} (v4)
    (v2) edge node [above] {\begin{color}{red}$0$\end{color}/$5$} (v1)
         edge node [right] {\begin{color}{red}$2$\end{color}/$2$} (v4)
         edge [bend left=75] node [right] {\begin{color}{red}$0$\end{color}/$3$} (v5)
    (v3) edge node [below] {\begin{color}{red}$0$\end{color}/$4$} (v2)
         edge node {\begin{color}{red}$0$\end{color}/$3$} (v4)
    (v4) edge node {\begin{color}{red}$0$\end{color}/$2$} (v5)
    (v5) edge node {\begin{color}{red}$4$\end{color}/$7$} (v3)
         edge [bend left=75] node {\begin{color}{red}$3$\end{color}/$6$} (v1)
    ;
    \end{tikzpicture}
    \caption{(a)}
    \label{fig:ex1a}
\end{figure}

In Figure (a), the $b$-flow is added in front of the capacities with red numbers


\subsection{case (b)}

\begin{figure}[h]
    \centering
    \begin {tikzpicture}[-latex, auto, node distance=25mm and 25mm ,on grid ,
                          semithick ,
                          main node/.style ={ circle , color=white ,
                          draw , black , text=black , minimum width =8mm}]
      \node[main node] (v1) [label=above:$3$]{$v_1$};
      \node[main node] (v2) [right=of v1, label=above:$1$] {$v_2$};
      \node[main node] (v3) [below=of v1, label=below:$2$] {$v_3$};
      \node[main node] (v4) [right=of v3, label=below:$2$] {$v_4$};
  \path[every node/.style={font=\sffamily\small}]
    (v1) edge node {$4$} (v3)
    (v2) edge node {$3$} (v1)
         edge node {$6$} (v4)
    (v3) edge node {$2$} (v2)
         edge node {$3$} (v4)
    ;
    \end{tikzpicture}
    \caption{(b)}
    \label{fig:ex1b}
\end{figure}

In figure (b) the nodes $v3$ and $v4$ can be seen as sources, each having an out-flow capacity of $2$, which sums to $4$. The node $v1$ demands $3$ and $v2$ demands $1$ which also sums to $4$, so far so good. If we make a cut seperating the graph in S=$\{v3,v4\}$ and T=$\{ v1, v2\}$, we can see that the summed capacity of the cut (in the direction S$\rightarrow$T) is $2$. This means that the maximum flow (max-flow-min-cut theorem) in the network is $2$, thus not fulfilling the demand of the sinks of $4$ units. Another way of realizing this is that $v4$ cannot send any of its generated surplus in any direction, thus not contributing in the network. 


\textbf{Max-flow-min-cut theorem: following statements are equivalent}
\begin{itemize}
  \item $f$ is a maximum flow in G
  \item The residual network $G_f$ contains no augmenting paths
  \item $|f| = c(S,T)$ for some cut $(S,T)$ of $G$
\end{itemize}

\textbf{}
